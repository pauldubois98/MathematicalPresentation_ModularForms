% !TeX spellcheck = en_GB
\begin{frame}
	\frametitle{Extra Slides and Links}
	\beamerbutton{\href{https://pauldubois98.github.io/HeckeOperatorsModuloTwo/}{Computations Results}}
	
	\beamerbutton{\href{https://pauldubois98.github.io/DirichletDensityTheoremAnimation/}{Dirichlet Density Theorem Animation}}
	
	\beamerbutton{\href{https://pauldubois98.github.io/HeckeOperatorsModuloTwo/GoverningFields/}{Governing Fields Found (Complete List)}}
	
	\beamerbutton{\href{https://pauldubois98.github.io/HeckeOperatorsModuloTwo/plot_code_to_int/}{Plots of Codes}}
	
	\beamerbutton{\href{https://en.wikipedia.org/wiki/Modular_forms_modulo_p}{Wikipedia Page on Modular Forms Modulo 2}}
	
\end{frame}

\begin{frame}
	\frametitle{Modularity}
	$$\mathbb{H} = \{z = x+yi \in \mathbb{C} | \ y>0 \}.$$
	We say that a function $f:\mathbb{H} \to \mathbb{C}$ is \textit{weakly modular} of \textit{weight} $2k$ if $f$ is meromorphic and
	$$
	f(z) = (cz+d)^{-2k} f \left( \frac{az+b}{cz+d} \right)
	\qquad \text{ for all }
	\begin{pmatrix} a & b\\
	c & d
	\end{pmatrix}
	\in \text{SL}_2(\mathbb{Z}).
	$$
	The group $\text{SL}_2(\mathbb{Z})$ of invertible 2-by-2 matrices over $\mathbb{Z}$ with  is generated by
	$$
	S = \begin{pmatrix} 0 & -1 \\
	1 &  0
	\end{pmatrix}
	\quad \text{ and } \quad
	T = \begin{pmatrix} 1 & 1 \\
	0 & 1 
	\end{pmatrix};
	$$
	$\text{SL}_2(\mathbb{Z})$ is called the modular group.
\end{frame}

\begin{frame}
	\frametitle{Eisenstein Series}
	Eisenstein Series:
	$$
	G_k(z) = \sum_{(m,n) \in \mathbb{Z}^2\setminus\{(0,0)\}} \frac{1}{(mz+n)^{2k}} \text{ for } k \geq 2
	$$
	Normalized Eisenstein Series:
	$$
	E_k \cdot 2\zeta(2k) = G_k,
	$$
	
	Discriminant Delta (before normalization):
	$$
	\varDelta = \left( \frac{1}{(2\pi)^{12}} \right) (g_2^3 - 27g_3^2) \in M_6
	$$
	\begin{flushright}
		$\text{ with } g_2 = 40G_2 \text{ and } g_3 = 140G_3$
	\end{flushright}
\end{frame}

\begin{frame}
	\frametitle{Subspaces of Modular Forms Modulo 2}
	$$
	\mathcal{F}_1
	= \left\langle \Delta^k \mid k = 1 \bmod 8 \right\rangle
	= \left\langle \Delta^1, \Delta^9, \Delta^{17}, \Delta^{25}, \dots \right\rangle
	$$
	$$
	\mathcal{F}_3
	= \left\langle \Delta^k \mid k = 3 \bmod 8 \right\rangle
	= \left\langle \Delta^3, \Delta^{11}, \Delta^{19}, \Delta^{27}, \dots \right\rangle
	$$
	$$
	\mathcal{F}_5
	= \left\langle \Delta^k \mid k = 5 \bmod 8 \right\rangle
	= \left\langle \Delta^5, \Delta^{13}, \Delta^{21}, \Delta^{29}, \dots \right\rangle
	$$
	$$
	\mathcal{F}_7
	= \left\langle \Delta^k \mid k = 7 \bmod 8 \right\rangle
	= \left\langle \Delta^7, \Delta^{15}, \Delta^{23}, \Delta^{31}, \dots \right\rangle
	$$
	
	$$
	\mathcal{F} = \mathcal{F}_1 \oplus \mathcal{F}_3 \oplus \mathcal{F}_5 \oplus \mathcal{F}_7
	$$
	
	\vspace{0.25cm}
	
	$$
	f \in \mathcal{F}_i \implies T_p|f \in \mathcal{F}_j
	\quad \text{ with } j \equiv pj \bmod 8
	$$
	
	\vspace{1cm}
\end{frame}

\begin{frame}
	\frametitle{Frobenius Element}
	With $L/K$ a Normal extension, and ideal $\mathfrak{P}$ in $\mathcal{O}_L$.
	$\text{Frob}_{L/K}(\mathfrak{P})$, and is \textit{the} element $\sigma \in \text{Gal}(L/K)$ such that:
	$$
	\sigma \mathfrak{P} = \mathfrak{P}
	\quad \text{ and } \quad
	\sigma(\alpha) \equiv \alpha^{\text{Norm}_{K/\mathbb{Q}}(\mathfrak{p})} \bmod{\mathfrak{P}} \quad \forall \alpha \in \mathcal{O}_L.
	$$
\end{frame}

\begin{frame}
	\frametitle{Densities}
	\textbf{Natural density:}
	We say that $S \subseteq \mathbb{P}$ has natural density $\delta$ when:
	$$
	\lim_{x \to +\infty}
	\frac{ \# \{ p \in \mathbb{P}, p < x \mid p \in S \}}
	{ \# \{ p \in \mathbb{P}, p < x \mid p \in \mathbb{P} \}} = \delta
	$$
	
	\vspace{0.5cm}
	
	\textbf{Analytic density or Dirichlet density:}
	We say that $S \subseteq \mathbb{P}$ has analytical (or Dirichlet) density $\delta$ when:
	$$
	\lim_{s \to 1^+}
	\left( \sum_{p \in S} \frac{1}{p^s} \right) 
	\left( \sum_{p \in \mathbb{P}} \frac{1}{p} \right)^{-1} = \delta
	$$
\end{frame}

\begin{frame}
	\frametitle{Project Summary}
	\begin{itemize}
		\item 1GB of data generated trough various computations
		\item 10 new very strong potential governing fields
		\item A Wikipedia article on Modular Forms Modulo 2
		%\item Project maths paper
		%\item Project presentation
	\end{itemize}
\end{frame}